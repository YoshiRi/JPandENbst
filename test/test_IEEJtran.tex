\documentclass[10pt,a4paper]{article}
\usepackage[utf8]{inputenc}
\usepackage{amsmath}
\usepackage{amsfonts}
\usepackage{amssymb}
\usepackage{graphicx}
\usepackage{listings, jlisting}
\author{JPandENbst}
\lstset{
  basicstyle={\scriptsize\ttfamily},
  identifierstyle={\small},
  commentstyle={\smallitshape},
  keywordstyle={\small\bfseries},
  ndkeywordstyle={\small},
  stringstyle={\small\ttfamily},
  frame={tb},
  breaklines=true,
  columns=[l]{fullflexible},
  numbers=left,
  xrightmargin=0zw,
  xleftmargin=0zw,
  numberstyle={\scriptsize},
  stepnumber=1,
  numbersep=0.5zw,
  lineskip=0.5ex
}
\begin{document}


\section{日本語文献}
\subsection{article\\}
japaneseTest1 \cite{japaneseTest1}, 
japaneseTest2 \cite{japaneseTest2}

\begin{lstlisting}
@article{japaneseTest1,
 author = {山田 一郎 and 山田 次郎 and 山田 三郎 and 山田 四郎},
 isjapanese = {true},
 year = {2019},
 journal = {日本語学会},
 title = {文献1},
 number = {10},
 pages = {20--30},
 volume = {15}
}
@article{japaneseTest2,
 author = {山田 五郎 and 山田 六郎},
 isjapanese = {true},
 year = {2019},
 journal = {日本語学会},
 title = {文献2},
 number = {10},
 pages = {21},
 volume = {15}
}
\end{lstlisting}

\subsection{book\\}
japaneseBook1 \cite{japaneseBook1}, 
japaneseBook2 \cite{japaneseBook2}

\begin{lstlisting}
@book{japaneseBook1,
 author = {佐藤 一郎},
 isjapanese = {true},
 year = {2010},
 publisher = {日本語出版},
 address = {東京},
 title = {日本語本},
 edition = {1},
 pages = {100--200}
}
@book{japaneseBook2,
 author = {佐藤 二郎 and 佐藤 三郎},
 isjapanese = {true},
 year = {2012},
 publisher = {日本語出版},
 address = {東京},
 title = {日本語本2},
 edition = {2},
 pages = {100--200}
}
\end{lstlisting}

\subsection{incollection\\}
japaneseBook3 \cite{japaneseBook3}, 
japaneseBook4 \cite{japaneseBook4}

\begin{lstlisting}
@incollection{japaneseBook3,
 author = {佐藤 二郎 and 佐藤 三郎},
 isjapanese = {true},
 year = {2012},
 publisher = {日本語出版},
 address = {東京},
 title = {日本語本2の中の抜粋},
 booktitle = {日本語本2},
 edition = {2},
 pages = {100--200}
}
@incollection{japaneseBook4,
 author = {山田 一郎},
 year = {1900},
 publisher = {出版会社},
 address = {東京},
 title = {タイトル},
 booktitle = {日本語本3},
 chapter = {3},
 edition = {1},
 pages = {100--150},
}
\end{lstlisting}


\section{英語}
\subsection{article}

englishTest1 \cite{englishTest1}, 
englishTest2 \cite{englishTest2}

\begin{lstlisting}
@article{englishTest1,
 author = {Ichiro Yamada and Jiro Yamada and Saburo Yamada and Shiro Yamada},
 year = {2019},
 journal = {IEEE Transactions on Pattern Analysis and Machine Intelligence},
 title = {Title1},
 number = {10},
 pages = {20--30},
 volume = {15},
}
@article{englishTest2,
 author = {Goro Yamada and Rokuro Yamada},
 year = {2019},
 journal = {IEEE Transactions on Pattern Analysis and Machine Intelligence},
 title = {Title2}
}
\end{lstlisting}

\subsection{book}
englishBook1 \cite{englishBook1}

\begin{lstlisting}
@book{englishBook1,
 address = {City, Country},
 publisher = {Publisher},
 author = {Engligh Author and German Author and Japanese Author},
 year = {2000},
 title = {Title, not booktitle},
 edition = {3},
 pages = {1-10},
}
\end{lstlisting}

\subsection{incollection}
englishBook2 \cite{englishBook2}

\begin{lstlisting}
@incollection{englishBook2,
 address = {City, Country},
 publisher = {Publisher},
 author = {Engligh Author and German Author and Japanese Author},
 booktitle = {Booktitle},
 chapter = {4},
 pages = {200--220},
 title = {Title},
 edition = {2},
 year = {2000},
}
\end{lstlisting}

% \bibliographystyle{../jIEEEtran}
% \bibliographystyle{../jIEEEtranS}
\bibliographystyle{../IEEJtran}

\bibliography{samplebib}
\end{document}